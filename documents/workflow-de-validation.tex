\documentclass[11pt,fleqn]{book} % Default font size and left-justified equations

\usepackage[top=3cm,bottom=3cm,left=3.2cm,right=3.2cm,headsep=10pt,letterpaper]{geometry} % Page margins

\usepackage{xcolor} % Required for specifying colors by name
\definecolor{ocre}{RGB}{52,177,201} % Define the orange color used for highlighting throughout the book
\usepackage{wallpaper}
\usepackage{mdframed}
\usepackage[top=2cm, bottom=2cm, outer=0cm, inner=0cm]{geometry}
% Font Settings
\usepackage{avant} % Use the Avantgarde font for headings
%\usepackage{times} % Use the Times font for headings
\usepackage{mathptmx} % Use the Adobe Times Roman as the default text font together with math symbols from the Sym­bol, Chancery and Computer Modern fonts
\graphicspath{ {figures/} }
\usepackage{array}
\usepackage[T1]{fontenc}
\usepackage{imakeidx}
\makeindex
\usepackage[totoc]{idxlayout}
\usepackage{tabularx}
\usepackage{caption}
\usepackage{microtype} % Slightly tweak font spacing for aesthetics
\usepackage[utf8]{inputenc} % Required for including letters with accents
\usepackage[T1]{fontenc} % Use 8-bit encoding that has 256 glyphs
\usepackage{hyperref}
% Bibliography
\usepackage[style=alphabetic,sorting=nyt,sortcites=true,autopunct=true,babel=hyphen,hyperref=true,abbreviate=false,backref=true,backend=biber]{biblatex}
\addbibresource{bibliography.bib} % BibTeX bibliography file
\defbibheading{bibempty}{}

\input{structure} % Insert the commands.tex file which contains the majority of the structure behind the template

\begin{document}
\title{Assurance de Qualite}

%----------------------------------------------------------
%	TITLE PAGE
%----------------------------------------------------------
\begingroup
\ThisLRCornerWallPaper{1.0}{Pictures/cover page.png}
\endgroup

%----------------------------------------Version page ---------

\newpage
~\vfill
\thispagestyle{empty}

%-------------------	TABLE OF CONTENTS

\pagestyle{empty} % No headers

\tableofcontents

\pagestyle{fancy} % Print headers again

%----------------------------------------------------
\chapter{Workflow de validation}
\section{Définition}
Un workflow ou circuit de validation permet de fluidifier la circulation de l’information et fiabiliser les processus internes. Donc c’est un ensemble d’étapes de niveaux d’approbation successifs, effectuées par différents intervenants.
\section{Objectifs d’un Workflow de Validation}
Le workflow de validation permet de :
\begin{itemize}
    \item Visualiser l’avancement en temps réel du document.
    \item Garantir le suivi, la sécurisation et la crédibilité des documents.
    \item Un meilleur contrôle de gestion de version des documents
    \item Un moyen plus fiable et plus rapide d’exécution des processus.
\end{itemize}
\section{Description du Workflow de Validation}
\begin{itemize}
    \item Le chef de projet prend en charge la répartition des taches sur les membres de l’équipe.
    \item Chaque tache doit être suivie d’une phase de validation (par toute l’équipe) et une vérification des normes (par le responsable de qualité).
    \item Après la collection des résultats des taches une validation sera effectuée suivie d’une vérification finale par le responsable de la qualité.
    \item Le chef de projet reçoit la version validée de document qui sera envoyée comme livrable final
\end{itemize}
\newpage
\section{WorkFlow De Validation}
Le schéma suivant résume ce workflow :
\begin{figure}[h]
    \centering
    \includegraphics[width=1\textwidth]{Pictures/WorkFlow.jpg}
    \caption{schéma du workflow de validation}
    \label{fig:pca}
\end{figure}
\end{document}
